\documentclass{scrartcl}

\usepackage{graphicx} % Required for the inclusion of images

\renewcommand{\labelenumi}{\alph{enumi}.}
% deutsche Sprache und Umlaute
\usepackage[utf8x]{inputenc}
\usepackage[ngerman]{babel}


%BibTex verweise
\usepackage{cite}

% für url pfade
\usepackage{url}

% setzt beim neuen Absatz das erste Wort 4 mm nach rechts \\ kleine Abstand vom Rand
\setlength{\parindent}{4mm}

%für boxen
\usepackage{pifont,mdframed}

\usepackage{microtype}

% hack für scr paket für Indexierung
\usepackage{scrhack}

% erzwingen der Positionierung
\usepackage{float}
\restylefloat{figure}

% Verbindungen vom Index, Tabellen und Abbildungen
\usepackage{hyperref}

%Tiefe des Inhaltsverzeichnis
\setcounter{tocdepth}{5}

%für Unterstriche
\usepackage{underscore}

% Beschriftung
\usepackage{caption}
\captionsetup{%
  font=small,
  labelfont=bf,
}

\usepackage{fancybox}

%\geometry{showframe}% for debugging purposes -- displays the margins

\usepackage{amsmath}
\usepackage{amssymb}
\usepackage{geometry}
\geometry{verbose,a4paper,tmargin=20mm,bmargin=25mm,lmargin=15mm,rmargin=20mm}
% Set up the images/graphics package

% The following package makes prettier tables.  We're all about the bling!
\usepackage{booktabs}

% The units package provides nice, non-stacked fractions and better spacing
% for units.
\usepackage{units}

% The fancyvrb package lets us customize the formatting of verbatim
% environments.  We use a slightly smaller font.
\usepackage{fancyvrb}
\fvset{fontsize=\normalsize}

% Small sections of multiple columns
\usepackage{multicol}

% Provides paragraphs of dummy text
\usepackage{lipsum}

% listing
\usepackage{listings}
\usepackage{color}
\definecolor{grey}{rgb}{0.4,0.4,0.4}
\definecolor{darkblue}{rgb}{0.0,0.0,0.6}
\definecolor{cyan}{rgb}{0.0,0.6,0.6}

\lstset{
  basicstyle=\tiny,
  columns=fullflexible,
  showstringspaces=false,
  commentstyle=\color{gray}\upshape
}

\lstdefinelanguage{XML} {
  morestring=[b]",
  morestring=[s]{>}{<},
  morecomment=[s]{<?}{?>},
  stringstyle=\color{black},
  identifierstyle=\color{darkblue},
  keywordstyle=\color{cyan},
  morekeywords={xmlns,version,type}
}

\makeatletter
\newenvironment{CenteredBox}{%
\begin{Sbox}}{% Save the content in a box
\end{Sbox}\centerline{\parbox{\wd\@Sbox}{\TheSbox}}}% And output it centered
\makeatother



\title{Rechnersicherheit - Übung 07}
\author{Martin Görick, Kai Kriedemann und Dennis Hägler \\ Tutor: Stefan Pfeiffer}
%\date{\today}  % if the \date{} command is left out, the current date will be used


\begin{document}
\maketitle


\section*{Aufgabe 1 - Seed-Generierung}

B und A,C,D sind offensichtlich unabhängig voneinander, weil dies zunächst 2 vollkommen unterschiedliche 
Systemaufrufe sind. Wichtig ist es, dass sie unabhängig voneinander sind, da man sonst aus Kenntnis dreier 
Werte den 4. berechnen könnte. Angenommen, A,C und D sind bekannt, B jedoch nicht. Wären sie nicht unabhängig, 
so wäre B jetzt berechenbar (abhängig). Sind sie unabhängig, so muss der Angreifer nur noch die $2^{23}$ des 
Wertes B ermitteln. Dies sind zwar immer noch sehr viele Werte, aber nicht unmöglich zu berechnen. Deshalb 
sollte eine Unabhängigkeit bestehen.
\newline

A,C und D sind zwar alles Aufrufe der selben Funktion, sind aber deshalb unabhängig voneinander, weil sie zu 
vollkommenen Zeitpunkten aufgerufen werden. Es kann nicht garantiert werden, dass diese jeweils in gleichen 
zeitlichen Abständen voneinander aufgerufen werden (es sei denn, man führt eine Automatisierung dieser 
Vorgänge ein). Je nach Auslastung und HW variieren also dort die zurückgekommenen Zeitpunkte und Werte. 
Man kann lediglich sagen, dass $A \leq C \leq D$, weil nach zeitlichem Ablauf wesentlich mehr 
Prozessorzyklen erfolgt sind.

\section*{Aufgabe 2 - Deterministische Zufallszahlengeneratoren}

\subsection*{a}

Zunächst hat man bei diesem Verfahren den Startwert Seed. Dieser dient als Input füt eine Hashfunktion. Das Ergebnis dessen ist wiederum Input für die nä. iterative Anwendung der Hashfunktion. Die in der VL kennen gerlernten krypto. Hashfunktionen sind dabei Einwegfunktionen. Dies bedeutet, für alle x H(x) effizient zu berechnen ist, für alle y jedoch ist es schwer, das m aus $H^{-1}$(y) zu finden. Da auch hierbei die Hashfunktion iterativ angewendet wird (Teilfolge ist Input für nä. Stufe), und diese Einwegfunktionen sind, ist es also nicht möglich, aus einer bel. Teilfolge Vor- bzw. Nachfolger berechnen zu können. Denn dann müsste die Umkehrung ebenfalls effizient berechenbar sein, was sie jedoch nicht ist.

\subsection*{b}

Hier kommen ein Initialisierungsvektor IV, ein Counter, ein Schlüssel K, Blockchiffren und Klartexte zum Einsatz. Input für die Blockchiffre ist eine Konkatenation des Ivs (Nonce) und des Counters (iterative Erhöhung), sowie der Schlüssel k. Das Ergebnis wird mit einem Klartextblock (KTB) addiert und dieses Ergebnis wiederum ist ein 
Geheimtextblock (GTB). Dieser Vorgang ist dabei komplett parallel durchführbar, d.h. es bestehen keine 
Abhängigkeiten untereinander. Dies bedeutet dabei, dass man also aus der Kenntnis eines Blockes nicht 
benachbarte oder andere Blöcke kompromittieren kann. Auf die Aufgabe bezogen bedeutet dies, dass man 
aus einer Teilfolge also nicht auf den Vorgänger bzw. Nachfolger schließen kann.

\section*{Aufgabe 3 - Instanzauthentisierung}

\subsection*{Passwortlisten}
\begin{itemize}
\item  Phishing: ja. Falsche Website gibt sich als richtige aus und gelangt an TAN. Angreifer muss nun hoffen/dafür sorgen, dass Bank gleiches i abfragt wie er den Benutzer (oder Akkumulation vieler i von Benutzer)
\item Key Logging: kriegt dann $p_{i}$ mit und somit erfolgreich. Verfällt ja aber nach Benutzung, deshalb eher suboptimal für Angreifer, da dieser en Transfer von $p_{i}$  abbrechen müsste, damit er es schicken kann
\item Replay: nach Ausgabe des i. Passwortes wird dieses ja von der Liste gestrichen $\Rightarrow$ Angriff nicht möglich
\end{itemize}
\subsection*{Zeitgesteuerte Passwortgeneratoren}
\begin{itemize}
\item  Phishing:  kriegt dann zwar das gesendete p, kennt jedoch das vorab ausgetauschte g nicht; schickt er aber
die Antwort immerhalb des Toleranz-Intervalls $\Rightarrow$  Angriff geglückt
\item Key Logging:  kriegt zwar dann die message mit, allerdings nicht den key, da der ja bereits vorab ausgetauscht worden ist
\item Replay: generell nicht; innerhalb des Toleranzintervalls jedoch schon (in VL: 30 s)
\end{itemize}
\subsection*{Ereignisgesteuerte Passwortgeneratoren}
\begin{itemize}
\item  Phishing: Angreifer kennt zwar g nicht; täuscht aber Opfer und kriegt pt (ohne Notwendigkeit der 
Kenntnis von S oder k); schickt  $p_{t}$ an Prüfer;  $p_{t}$ ist ja gültiger Lamport-Hash $\Rightarrow$  
akzeptiert $\Rightarrow$  getäuscht
\item Key Logging:  nach Benutzung eines Einmalpasswortes wird ein anderes benutzt; Keylogger kann dann zwar das aktuelle Passwort mit, müsste aber Sendung des aktuellen Passwortes vom Opfer verhindern und selbst die Sendung vornehmen; also eher wirkungslos
\item Replay:  Generierung neuer Einmalpasswörter (N Stück); bei vollständiger Benutzung: neuer Startwert 
S für neue N Hashfunktionen; Replay-A. somit nicht möglich
\end{itemize}
\subsection*{Challenge Response Passwortgeneratoren}
\begin{itemize}
\item  Phishing: Angreifer kennt k nicht, kriegt dann c übertragen, schickt c an das Opfer, bekommt das 
r, schickt r an den Prüfer, r = r' $\Rightarrow$  ohne Kenntnis von k erfolgreich getäuscht
\item Key Logging: loggt dann r mit (falls es eingegeben wird), aber auch hier wieder: müsste Übertragung 
von r an Prüfer durch Beweisenden verhindern und stattdessen sein geloggtes r schicken
\item Replay: wenn c wirklich jedes Mal wechselt, ändert sich auch r; Angreifer müsste dann sehr, sehr 
viele c sammeln; theoretisch möglich, praktisch wenig rentabel
\end{itemize}

\section*{Aufgabe 4 - Passwörter}

Einmalpasswörter schützen gegen Ausspähen laut Skript. Problematisch: Man-in-the-Middle.
\newline

Steht ja eigentlich im Skript RS13, Seite 60 unten/61 oben –  z.B.Verfahren, die eine gegenseitige 
Authentisierung beinhalten.
\newline

Einmalpasswörter basieren auf einer 1-Faktor-Authentisierung. Also fügt man bei den Verfahren mind. einen 2. Faktor hinzu.

\end{document}
